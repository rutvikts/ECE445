\documentclass{report}

\usepackage{amsmath}
\usepackage{graphicx} % to insert images
\graphicspath{{../pdf/}{/Users/rutviksayankar/Repos/ECE445_SP24/TACheckIn/Rutvik/IndividualProgressReport/Images}} % setting up path for image storage
\usepackage{hyperref} % allows table of contents to be clickable to sections
\hypersetup{
    colorlinks,
    citecolor=blue,
    filecolor=blue,
    linkcolor=blue,
    urlcolor=blue
}

\title{ECE 445: Senior Design Lab \\ Final Report \\ Oxygen Delivery Robot} % Sets article title

\author {
    \textbf{Team 27} \\ 
    \textit{Sayankar, Rutvik}\\
    \textit{rutviks2@illinois.edu} \\
    \hfill \\ 
    \textit{Dunican, Aidan}\\
    \textit{dunican2@illinois.edu} \\
    \hfill \\ 
    \textit{Kalyniouk, Nazar}\\
    \textit{nazark2@illinois.edu} \\
    \hfill \\ 
    \textbf{Teaching Assistant} \\ 
    \textit{Subramaniam, Selva} \\
    \textit{ss170@illinois.edu} 
} 
\date{\today} % Sets date for date compiled

% This section redefines the chapter headings, this allows us to remove the implicit "Chapter #" that comes before a provided chapter title.
\makeatletter
\def\@makechapterhead#1{%
  \vspace*{50\p@}%
  {\parindent \z@ \raggedright \normalfont
    \ifnum \c@secnumdepth >\m@ne
      %\if@mainmatter
        %\huge\bfseries \@chapapp\space \thechapter
        \Huge\bfseries \thechapter.\space%
        %\par\nobreak
        %\vskip 20\p@
      %\fi
    \fi
    \interlinepenalty\@M
    \Huge \bfseries #1\par\nobreak
    \vskip 40\p@
  }}
\makeatother

% The preamble ends with the command \begin{document}
\begin{document}
    \maketitle % creates title using information in preamble (title, author, date)

    \begin{abstract}
        Test
    \end{abstract}
    
    \pagebreak
    \tableofcontents % creates table of contents
    \pagebreak

    \chapter{Introduction}
    \section{Project Overview}
    Children's interstitial and diffuse lung disease (ChILD) is a collection of diseases or disorders. These diseases cause a thickening of the interstitium (the tissue that extends throughout the lungs) due to scarring, inflammation, or fluid buildup \cite{ChILD-2022}. This eventually affects a patient’s ability to breathe and distribute enough oxygen to the blood. Numerous children experience the impact of this situation, requiring supplemental oxygen for their daily activities. It hampers the mobility and freedom of young infants, diminishing their growth and confidence. Moreover, parents face an increased burden, not only caring for their child but also having to be directly involved in managing the oxygen tank as their child moves around.

    Given the absence of relevant solutions in the current market, our project aims to ease the challenges faced by parents and provide the freedom for young children to explore their surroundings. As a proof of concept for an affordable solution, we propose a three-wheeled omnidirectional mobile robot capable of supporting a filled M-6 oxygen tank. We plan to implement two localization subsystems to ensure redundancy and enhance child safety. The first subsystem utilizes ultra-wide band (UWB) transceivers for triangulating the child's location relative to the robot in indoor environments, this is similar to how Apple AirTags triangulate their location relative to an iPhone \cite{airtag_uwb} (although AirTags use a combination of UWB and Bluetooth triangulation \cite{airtag_ble}). The second subsystem makes use of a desktop web camera which streams video to a Raspberry Pi where it will leverage open-source object tracking libraries to improve our directional accuracy in tracking a child. The final main subsystem focuses on the drive train, chassis and close range object detection of the robot.

    \section{Block Diagram}
    insert block diagram here 

    \subsection{Omniwheel Drivetrain System}
    General Overview on system here

    \subsection{Ultra Wideband Triangulation System}
    General Overview on system here
    
    \subsection{Close Range Object Detection System}
    General Overview on system here
    
    \subsection{Computer Vision System}
    General Overview on system here
    
    \subsection{Control System}
    General Overview on system here
    
    \subsection{Power System}
    General Overview on system here


    \chapter{Design}
    \section{Design Procedure}

    \subsection{Omniwheel Drivetrain System}
    Discuss your design decisions for each block at the most general level: What alternative approaches to the design are possible, which was chosen, and why is it desirable? Introduce the major design equations or other design tools used; show the general form of the circuits and describe their functions.

    \subsection{Ultra Wideband Triangulation System}
    Discuss your design decisions for each block at the most general level: What alternative approaches to the design are possible, which was chosen, and why is it desirable? Introduce the major design equations or other design tools used; show the general form of the circuits and describe their functions.
    
    \subsection{Close Range Object Detection System}
    Discuss your design decisions for each block at the most general level: What alternative approaches to the design are possible, which was chosen, and why is it desirable? Introduce the major design equations or other design tools used; show the general form of the circuits and describe their functions.
    
    \subsection{Computer Vision System}
    Discuss your design decisions for each block at the most general level: What alternative approaches to the design are possible, which was chosen, and why is it desirable? Introduce the major design equations or other design tools used; show the general form of the circuits and describe their functions.
    
    \subsection{Control System}
    Discuss your design decisions for each block at the most general level: What alternative approaches to the design are possible, which was chosen, and why is it desirable? Introduce the major design equations or other design tools used; show the general form of the circuits and describe their functions.
    
    \subsection{Power System}
    Discuss your design decisions for each block at the most general level: What alternative approaches to the design are possible, which was chosen, and why is it desirable? Introduce the major design equations or other design tools used; show the general form of the circuits and describe their functions.

    \section{Design Details}

    \subsection{Omniwheel Drivetrain System}
    Present the detailed design, with diagrams and component values. Show how the design equations were applied. Give equations and diagrams with specific design values and data. Place large data tables in an appendix. Circuit diagrams that are too large to be readable on a single page should be broken into pieces for presentation. The full diagram may be included in an appendix. Use photographs only as necessary and treat them, along with all other graphics except tables, as figures.

    \subsection{Ultra Wideband Triangulation System}
    Present the detailed design, with diagrams and component values. Show how the design equations were applied. Give equations and diagrams with specific design values and data. Place large data tables in an appendix. Circuit diagrams that are too large to be readable on a single page should be broken into pieces for presentation. The full diagram may be included in an appendix. Use photographs only as necessary and treat them, along with all other graphics except tables, as figures.

    \subsection{Close Range Object Detection System}
    Present the detailed design, with diagrams and component values. Show how the design equations were applied. Give equations and diagrams with specific design values and data. Place large data tables in an appendix. Circuit diagrams that are too large to be readable on a single page should be broken into pieces for presentation. The full diagram may be included in an appendix. Use photographs only as necessary and treat them, along with all other graphics except tables, as figures.

    \subsection{Computer Vision System}
    Present the detailed design, with diagrams and component values. Show how the design equations were applied. Give equations and diagrams with specific design values and data. Place large data tables in an appendix. Circuit diagrams that are too large to be readable on a single page should be broken into pieces for presentation. The full diagram may be included in an appendix. Use photographs only as necessary and treat them, along with all other graphics except tables, as figures.

    \subsection{Control System}
    Present the detailed design, with diagrams and component values. Show how the design equations were applied. Give equations and diagrams with specific design values and data. Place large data tables in an appendix. Circuit diagrams that are too large to be readable on a single page should be broken into pieces for presentation. The full diagram may be included in an appendix. Use photographs only as necessary and treat them, along with all other graphics except tables, as figures.  

    \subsection{Power System}
    Present the detailed design, with diagrams and component values. Show how the design equations were applied. Give equations and diagrams with specific design values and data. Place large data tables in an appendix. Circuit diagrams that are too large to be readable on a single page should be broken into pieces for presentation. The full diagram may be included in an appendix. Use photographs only as necessary and treat them, along with all other graphics except tables, as figures.

    \chapter{Verification}

    Discuss the testing of the completed project and its major blocks. Provide solid technical data, and present it in an easily grasped manner, using graphs where necessary. Include any standard tests for your type of circuit and all specific ones you feel are needed to prove that the design goals were met. Discuss the Requirement and Verification Table from your design review. Including the table in an appendix will help avoid lengthy and tedious narrative description in the main text, which may not be of immediate interest to your imagined audience of managers. Do not discuss low-level requirements unless they failed to verify, or you found that they were critical in some unexpected way, or you need to makes changes—for instance, to the tolerances or acceptable ranges of quantitative results. It is important to hit the main points and explain any requirement that is not verified, but keep the discussion concise and refer interested readers to the appendix for details.

    \section{Omniwheel Drivetrain System}
    Verification details on the system here

    \section{Ultra Wideband Triangulation System}
    Verification details on the system here
    
    \section{Close Range Object Detection System}
    Verification details on the system here
    
    \section{Computer Vision System}
    Verification details on the system here
    
    \section{Control System}
    Verification details on the system here
    
    \section{Power System}
    Verification details on the system here

    \chapter{Costs}

    Labor cost estimates should use the following formula for each partner: ideal salary (hourly rate) = actual hours spent * 2.5 
    
    Include estimates for electronics and machine shop hours, as applicable. For parts, use real values when you know them; make realistic estimates otherwise. List both the retail cost and what you or the department paid (in this case you may list lab-owned pieces as free). If the project might be commercially viable, estimate the cost of mass-production by listing bulk-purchase costs. Make sure any tables are numbered appropriately, given titles, and cited directly in the text. 

    \chapter{Conclusions}

    Bring together, concisely, the conclusions to be drawn. It may be appropriate, depending on the nature of the project, to begin or end with a two- or three-sentence executive summary. The reader needs to be convinced that the design will work. Summarize your accomplishments. If uncertainties remain, they should be pointed out, and alternatives, such as modifying performance specifications, should be spelled out to deal with foreseeable outcomes. Use words, not equations or diagrams. Devote a section to ethical considerations with reference to the IEEE Code of Ethics and any other applicable code (e.g., the AMA Code of Medical Ethics for certain bioengineering projects). Either here or in the background discussion of your introduction, provide a paragraph addressing the broader impacts of your project in terms of global, economic, environmental and/or societal contexts.

    \section{Ethics and Safety}
    \subsection{Ethical Considerations}
    As we continue through the development of our project, we are unwavering in our dedication to abide by the ethical and safety principles outlined by the Association for Computing Machinery (ACM) and the Institute of Electrical and Electronics Engineers (IEEE). As we embark on this project, we pledge our commitment to adhere to these standards, ensuring that our actions and choices uphold the highest level of professionalism and integrity.

    As outlined in Section I of the IEEE Code of Ethics, we pledge to “uphold the highest standards of integrity, responsible behavior, and ethical conduct in professional activities” \cite{IEEE_2020}. We will prioritize safety in our design and adhere to ethical design practices. Educating the parents and caregivers about the robot’s use and limitations will allow for informed decision-making. Following relevant laws and regulations regarding this technology will also be a high priority.

    In the same Code of Ethics, outlined in Section III, we pledge to “strive to ensure this code is upheld by colleagues and co-workers” \cite{IEEE_2020}. We will support each other in ethical conduct and foster a culture of ethical behavior. Open communication will be established and encouraged to raise concerns and provide guidance to team members.

    \subsection{Safety Considerations}
    This project aligns with the safety principles outlined in the ACM Code of Ethics and Professional Conduct. Safety remains our number one priority and as outlined in Section 1.2 \cite{ACM_2018}, we will avoid negative consequences, especially when those consequences are significant and unjust. We will take careful consideration of potential impacts and minimize harm. In the context of this project, we will ensure that the robot’s design and operation prioritizes safety, especially to the children this product aims to assist. We will work to analyze potential risk and consider the robot’s mobility and interaction with its environment.

    As well as promoting safety, privacy is also a very important guideline that will be followed, which is outlined in Section 1.6. As professionals, we must safeguard the personal information of our users, especially if it involves children. As it relates to this project, we will ensure that no data will be collected and stored in an external location. Data collection will be minimized to only what is necessary for the robot to operate.

    One aspect of our project where safety must be considered is regarding the use of lithium batteries. We acknowledge the potential risks associated with the misuse of lithium batteries. We are committed to following the safety guidelines associated with the batteries we plan on using. More specifically, maintaining the battery’s temperature within the recommended range. Also, we are dedicated to the responsible disposal of batteries to ensure sustainability.

    Since we are incorporating motors into our design, we will deploy essential control systems to mitigate potential hazards such as collisions with the environment. Safe operation will be ensured with the use of sensors, vision systems, and warning systems.

    \appendix

    \chapter{Appendix}

    \section{Terms and Keywords}
    \begin{itemize}
        \item ChILD - Children's Interstitial and Diffuse Lung Disease
        \item PCB – Printed Circuit Board
        \item OWB - Three-Wheeled Omni-Wheel Bot
        \item UWB - Ultra-wideband is a radio technology that can use a very low energy level for short-range, high-bandwidth communications over a large portion of the radio spectrum.
        \item ESP32 - A series of low-cost, low-power system-on-a-chip microcontrollers with integrated Wi-Fi and dual-mode Bluetooth.
        \item RPM - Revolutions per minute
        \item GPIO - General-purpose input/output
        \item SPI - Serial Peripheral Interface
        \item PWM - Pulse-width modulation
        \item MPH - Miles per hour
        \item OpenCV - (Open Source Computer Vision Library) is an open source computer vision and machine learning software library.
        \item TDoA - Time Difference of Arrival 
        \item ToF - Time of Flight
    \end{itemize}

    \bibliographystyle{IEEEtran}
    \bibliography{ref}

\end{document} % This is the end of the document