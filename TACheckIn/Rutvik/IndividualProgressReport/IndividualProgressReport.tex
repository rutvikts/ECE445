\documentclass{report}

\usepackage{amsmath}
\usepackage{hyperref} % allows table of contents to be clickable to sections
\hypersetup{
    colorlinks,
    citecolor=blue,
    filecolor=blue,
    linkcolor=blue,
    urlcolor=blue
}

\title{ECE 445: Senior Design Lab \\ Individual Progress Report} % Sets article title
\author{Rutvik Sayankar} % Sets authors name
\date{\today} % Sets date for date compiled

% The preamble ends with the command \begin{document}
\begin{document}
    \maketitle % creates title using information in preamble (title, author, date)
    
    \pagebreak
    \tableofcontents % creates table of contents
    \pagebreak

    \section{Terms and Keywords}
    \vspace{0.8cm}
    \begin{itemize}
        \item ChILD - Children's Interstitial and Diffuse Lung Disease
        \item PCB – Printed Circuit Board
        \item OWB - Three-Wheeled Omni-Wheel Bot
        \item UWB - Ultra-wideband is a radio technology that can use a very low energy level for short-range, high-bandwidth communications over a large portion of the radio spectrum.
        \item ESP32 - A series of low-cost, low-power system-on-a-chip microcontrollers with integrated Wi-Fi and dual-mode Bluetooth.
        \item RPM - Revolutions per minute
        \item GPIO - General-purpose input/output
        \item PWM - Pulse-width modulation
        \item MPH - Miles per hour
        \item OpenCV - (Open Source Computer Vision Library) is an open source computer vision and machine learning software library.
        \item OpenPose - Real-time multi-person keypoint detection library for body, face, hands, and foot estimation.
        \item TDoA - Time Difference of Arrival 
        \item ToF - Time of Flight
    \end{itemize}

    \pagebreak

    \chapter{Introduction}
    \section{Team Project Overview}
    Children's interstitial and diffuse lung disease (ChILD) is a collection of diseases or disorders. These diseases cause a thickening of the interstitium (the tissue that extends throughout the lungs) due to scarring, inflammation, or fluid buildup \cite{ChILD-2022}. This eventually affects a patient’s ability to breathe and distribute enough oxygen to the blood. Numerous children experience the impact of this situation, requiring supplemental oxygen for their daily activities. It hampers the mobility and freedom of young infants, diminishing their growth and confidence. Moreover, parents face an increased burden, not only caring for their child but also having to be directly involved in managing the oxygen tank as their child moves around.

    Given the absence of relevant solutions in the current market, our project aims to ease the challenges faced by parents and provide the freedom for young children to explore their surroundings. As a proof of concept for an affordable solution, we propose a three-wheeled omnidirectional mobile robot capable of supporting a filled M-6 oxygen tank. We plan to implement two localization subsystems to ensure redundancy and enhance child safety. The first subsystem utilizes ultra-wide band (UWB) transceivers for triangulating the child's location relative to the robot in indoor environments, this is similar to how Apple AirTags triangulate their location relative to an iPhone \cite{airtag_uwb} (although AirTags use a combination of UWB and Bluetooth triangulation \cite{airtag_ble}). The second subsystem makes use of a desktop web camera which streams video to a Raspberry Pi where it will leverage open-source object tracking libraries to improve our directional accuracy in tracking a child. The final main subsystem focuses on the drive train, chassis and close range object detection of the robot.

    \section{Individual Responsibilities}
    My individual responsibilities are as follows:

    \begin{itemize}
        \item Overall project management
        \item Research, design and implementation of OWB triangulation system
        \item Design and Routing of PCB
        \item Aid in chassis design
        \item Aid in bumper limit switch design
        \item Handle all 3D printing
    \end{itemize}

    \chapter{Individual Design Work}
    \section{Design Considerations}

    \section{Diagrams}

    \section{Testing/Verification}

    \chapter{Conclusion}

    \section{Self-assessment}

    \section{Plans for remaining work}

    
    \textbf{Hello World!} Today I am learning \LaTeX. %notice how the command will end at the first non-alphabet charecter such as the . after \LaTeX
     \LaTeX{} is a great program for writing math. I can write in line math such as $a^2+b^2=c^2$ %$ tells LaTexX to compile as math
     . I can also give equations their own space: \cite{IEEE_2020}
    \begin{equation} % Creates an equation environment and is compiled as math
    \gamma^2+\theta^2=\omega^2 
    \end{equation}
    If I do not leave any blank lines \LaTeX{} will continue  this text without making it into a new paragraph.  Notice how there was no indentation in the text after equation (1).  
    Also notice how even though I hit enter after that sentence and here $\downarrow$
     \LaTeX{} formats the sentence without any break.  Also   look  how      it   doesn't     matter          how    many  spaces     I put     between       my    words.
    
    For a new paragraph I can leave a blank space in my code. 


    \bibliographystyle{IEEEtran}
    \bibliography{ref}

\end{document} % This is the end of the document